\documentclass{article}
\input{../../preamble.tex}
\usepackage[letterpaper, portrait, margin=1in]{geometry}

\graphicspath{./figures}

\pagestyle{fancy}
\fancyhf{}
\lhead{Sidharth Baskaran}
\chead{Project Euler 114-117}
\rhead{04/18/2022}

\lstset{
	basicstyle=\ttfamily\small,
	breaklines=true,
	numberstyle=\color{gray},
    stringstyle=\color[HTML]{933797},
    commentstyle=\color[HTML]{228B22},
    emph={[2]from,import,pass,return}, emphstyle={[2]\color[HTML]{DD52F0}},
    emph={[3]range}, emphstyle={[3]\color[HTML]{D17032}},
    emph={[4]for,in,def}, emphstyle={[4]\color{blue}}
}

\begin{document}
    
\section{Problem Statement}

Problems 114-117 are variations of the same idea.
We are given a row with $n$ units and are asked to find how many ways
in which we can fit smaller blocks of varying size inside.
\begin{itemize}
	\item Problem 114 allows us to place blocks of any size, with a minimum length of 3, where multiple blocks are separated by a blank unit.
	\item Problem 115 defines a function $F(m,n)$ which returns the number of ways we can fit a block of size $m$ in $n$ units.
	We want to find the smallest $n$ such that $F(50,n)\geq 1,000,000$.
	\item Problem 116 defines block lengths of $m =2, 3, 4$, asking for $\sum_{i=1}^3F(m_i,50)$ (blocks cannot be mixed).
	\item Problem 117 is an extension of 116 where we can mix blocks of sizes $m=2,3,4$.
\end{itemize}

\section{Code}

The modified DP solutions are found \href{https://github.com/sidnb13/project-euler/blob/master/code/114-dp.py}{here}.

\section{General Approach}

It is actually easiest to start with an approach to Problem 114 and modify the solution for the other variations.
Using the constraints of Problem 114, note that the minimum block size is 3.
Let us define a function $f(n)$ such that $f(3)=1$; there is 1 way to fill a row
of size 3:

\begin{center}
	\begin{tabular}{|c|c|c|}
		\hline
		\cellcolor{red}&\cellcolor{red}&\cellcolor{red}\\
		\hline
	\end{tabular}
\end{center}

Symmetry does not matter in different cases, so for $n=4$ we have three ways:

\begin{center}
	\begin{tabular}{ccc}
		\begin{tabular}{|c|c|c|c|}
			\hline
			\cellcolor{red}&\cellcolor{red}&\cellcolor{red}&\\
			\hline
		\end{tabular}
		&
		\begin{tabular}{|c|c|c|c|}
			\hline
			\cellcolor{red}&\cellcolor{red}&\cellcolor{red}&\cellcolor{red}\\
			\hline
		\end{tabular}
		&
		\begin{tabular}{|c|c|c|c|}
			\hline
			&\cellcolor{red}&\cellcolor{red}&\cellcolor{red}\\
			\hline
		\end{tabular}
	\end{tabular}
\end{center}

We can generalize this to $n$ units. However, when we have $n>4$
there arises the case of the 1-unit separation between multiple blocks in a row.
Take the case of $n=7$. With a \textit{left offset} of 0, we can introduce the possibility
of filling the row with a block of size 6, and this option is unique to $n=7$. Namely, starting
with a minimum block size of 3 and left offset of 0, we have $7-3+1=5$ ways of placing single blocks of valid size.

The dynamic programming concept arises when we see that there are $n-k-1$ spaces left
to place other blocks when we have placed a block of size $k$ with left offset of 0.
In the case of $n=7$ once we place a block of say size 3, we can place another one of size 3 where
there is a single space in between; there are $7-3-1=3$ spaces left to fill with other blocks considering
the padding constraint. We know there is only one such way to fill 3 spaces, this is $f(3)$.

\begin{center}
	\begin{tabular}{|c|c|c|c|c|c|c|}
		\hline
		\cellcolor{red}&\cellcolor{red}&\cellcolor{red}&&\cellcolor{red}&\cellcolor{red}&\cellcolor{red}\\
		\hline
	\end{tabular}
\end{center}

Overall it is easy to see we are approaching $1 + f(a)$ where $a<n$, i.e. we place one block and then consider the subproblem
already solved in a previous iteration. We must loop this $1+f(a)$ over the possible offsets and block size combinations given a certain $n$ too.
Once we want to place another block, we have a new row size of $n-k-1$ to work with
where $k$ is in the range of block sizes we can accomodate in $n$, or $[3,n]$. This is because for each block we place
in a row of size $n$ we can place blocks of size $3,4,\ldots,n$. We must also consider the possible left offsets,
which are in the range $[1, n-3]$ since we already include the case of the offset 0.
Thus, our general formula becomes
\begin{align*}
	f_{114}(n)&=\underbrace{\sum_{k=3}^n (f(n-k-1)+1)}_{i=0} + \sum_{i=1}^{n-3}\left(\sum_{k=3}^{n-i} (f(n-k-i-1) + 1)\right)\\
	&=\sum_{i=0}^{n-3}\left(\sum_{k=3}^{n-i} \left(f(n-k-i-1) + 1\right)\right)
\end{align*}

This general formula gives a solution to 114, which allows for any possible block size and combination. Pseudocode is expressed in Pythonic syntax below.
\begin{lstlisting}
def f(n):
	dp = [0] * (n + 1) # our DP array
	dp[3] = 1 # base case
	for ni in range(4, n + 1):
		dp[ni] = sum(
			[ 
				sum([dp[ni-k-i-1]+1 for i in range(0, ni-k + 1)])

				for k in range(3, ni-i + 1) 
			]
		)
	return dp[n]
\end{lstlisting}
115 defines an $f(m,n)$ where we desire only to fit a block of size $m$. It is easy to modify our prevous $f$:
\begin{equation*}
	f_{115}(m,n)=\sum_{i=0}^{n-m}\left(\sum_{k=m}^{n-i} \left(f(n-k-i-1) + 1\right)\right)
\end{equation*}
since we just change the minimum block size from 3 to $m$.
We must implement a modified function $f(m,n)$ for Problem 116, which defines a singular
block size $m$ used to calculate the possibilities. There is no 1-block padding as well.
\begin{equation*}
	f_{116}(m,n)=\sum_{i=0}^{n-m}\left((f(n-m-i) + 1\right)
\end{equation*}

Problem 117 is really a modification of 115 but without a 1-block padding.
\begin{equation*}
	f_{117}(n) = \sum_{i=0}^{n-2}\left(\sum_{k=2}^{4} \left(f(n-k-i) + 1\right)\right)
\end{equation*}

\end{document}