\documentclass{article}
\input{../../preamble.tex}
\usepackage[letterpaper, portrait, margin=1in]{geometry}

\graphicspath{./figures}

\pagestyle{fancy}
\fancyhf{}
\lhead{Sidharth Baskaran}
\chead{Project Euler 7}
\rhead{01/08/2022}

\lstset{
	basicstyle=\ttfamily,
	frame=single,
	breaklines=true,
	mathescape
}

\begin{document}
    
\section{Problem Statement}

By listing the first six prime numbers: 2, 3, 5, 7, 11, and 13, we can see that the 6th prime is 13. What is the 10001st prime number?

\section{Code}

\lstinputlisting[language=Python]{../code/7.py}

\section{Explanation}

We include a method \verb|is_prime| to determine whether or not a number $n$ is prime.
It simply checks whether all numbers below $\sqrt{n}$ excluding $1$ divide $n$. We only have to check from $\sqrt{n}$ and below because at least one factor of a nonprime number is below its square root.

The method \verb|solve| finds the $n$th prime, with a default of $n=10001$. A prime variable, $p$, is set to 2
and a counter variable is set to 1 (since 2 is prime). We then check whether the next odd $p$ is still prime, and move our counter up until it is equal to 10001.
Only checking odd numbers after 2 saves time, since there are no even primes past 2.

\begin{figure}[H]
\centering
\begin{BVerbatim*}
104759
396.011ms
\end{BVerbatim*}
\caption{Output}
\end{figure}

\end{document}